% research-report.tex: 调研报告
% Copyright (C) 2022 吴骏东, 张子辰, 蓝俊玮, 郭耸霄 and 陈建绿
% All rights reserved.
\documentclass[UTF8,fontset=none,linespread=1.15]{ctexart}
\ctexset
{
    section/format={\Large\sffamily\bfseries},
    subsection/format+={\sffamily},
    subsubsection/format+={\itshape}
}
\setCJKmainfont[ItalicFont={KaiTi},BoldItalicFont={KaiTi},
BoldItalicFeatures={FakeBold=3}]{Noto Serif CJK SC}
\setCJKsansfont[AutoFakeSlant]{Noto Sans CJK SC DemiLight}
\setCJKmonofont[AutoFakeBold=3,AutoFakeSlant]{FangSong}
\setmainfont{cmun}[Extension=.otf,UprightFont=*rm,
ItalicFont=*ti,BoldFont=*bx,BoldItalicFont=*bi]
\setsansfont{cmun}[Extension=.otf,UprightFont=*ss,
ItalicFont=*si,BoldFont=*sx,BoldItalicFont=*so]
\setmonofont{cmun}[Extension=.otf,UprightFont=*btl,
ItalicFont=*bto,BoldFont=*tb,BoldItalicFont=*tx]
%Computer Modern Unicode 的\textasciitilde和\~{}的高度相同,所以用\tildechar表示居中的波浪线~
\newcommand{\tildechar}{\raisebox{-0.35em}{\textasciitilde}}
\usepackage[a4paper,hmargin=1.2in,vmargin=1in]{geometry}
\usepackage{graphicx,tikz,float,subfig,multicol,makecell,multirow,longtable}
\usepackage[normalem]{ulem}
\usepackage{CJKfntef}
\usepackage[perpage]{footmisc}

%目录, 参考了OSH-2021/x-sBPF
\usepackage{titletoc}
\titlecontents{section}[2em]{\addvspace{1.3mm}\bfseries}{%
\contentslabel{2.0em}}{}{\titlerule*[5pt]{$\cdot$}\contentspage}
\titlecontents{subsection}[4.2em]{}{\contentslabel{2.5em}}{}{%
\titlerule*[5pt]{$\cdot$}\contentspage}
\titlecontents{subsubsection}[7.2em]{}{\contentslabel{3.3em}}{}{%
\titlerule*[5pt]{$\cdot$}\contentspage}

%代码环境
\usepackage{listings}
\lstset{basicstyle={\normalfont\ttfamily},breaklines,tabsize=4}

\usepackage{enumitem}
\setlistdepth{5}
\renewlist{enumerate}{enumerate}{5}
\setlist{itemsep=0pt,partopsep=0pt,parsep=0pt,topsep=0pt}
\setlist[enumerate,1]{label=\arabic*.}
\setlist[enumerate,2]{label=(\arabic*)}
\setlist[enumerate,3]{label=\textcircled{\arabic*}}
\setlist[enumerate,4]{label=(\textit{\roman*})}
\setlist[enumerate,5]{label=\textit{\alph*})}

\usepackage[colorlinks,unicode,pdfstartview={FitH}]{hyperref}
\hypersetup
{
  pdftitle={2022春 操作系统原理与设计(H) x-runikraft小组 调研报告},
  pdfauthor={吴骏东; 张子辰; 蓝俊玮; 郭耸霄; 陈建绿}
}

%带圈数字,它必须在hyperref之后载入
\usepackage{xunicode-addon}
\makeatletter
\xeCJKDeclareCharClass{Default}{"24EA, "2460->"2473, "3251->"32BF,"24B6->"24E9,"2160->"217F}
\newfontfamily\EnclosedNumbers{Noto Serif CJK SC}
\AtBeginUTFCommand[\textcircled]{\begingroup\EnclosedNumbers}
\AtEndUTFCommand[\textcircled]{\endgroup}
\makeatother

\makeatletter
\let\textcircled@old\textcircled
\protected\def\textcircled#1{%
	\expandafter\textcircled@old\expandafter{\expanded{#1}}}
\makeatother
\makeatletter
\renewcommand\@makefntext[1]{%
	\setlength\parindent{0.75\ccwd}\selectfont
	\@thefnmark\ #1}
\makeatother
\renewcommand*\thefootnote{\textcircled{\arabic{footnote}}}
\renewcommand{\lstlistingname}{代码}

%\renewcommand{\today}{2022年3月27日}

\begin{document}
\sffamily %为方便屏幕阅读,文档主要使用无衬线字体
\title{\bfseries x-runikraft小组\quad 调研报告}
\author{吴骏东\and 张子辰\and 蓝俊玮\and 郭耸霄\and 陈建绿}
\date{\today}
\maketitle

\tableofcontents

\section{项目简介}
\textbf{本项目参考 Unikraft 的设计,用 Rust 语言实现模块化的 unikernel
在保持 Unikraft 的 POSIX 兼容性、可定制性的基础上,用 Rust 语言增强
内核的安全性。}

Unikernel是专一用途的、单地址空间的轻量操作系统。Unikernels在虚拟机
上运行时,能够提供比传统的容器更短的启动时间、更高的运行效率和更强的
隔离性,因此unikernels通常被用在云计算领域。然而,为了追求轻量性,
unikernels裁剪了传统的操作系统的众多组件,因此unikernels无法提供许多
常用的库的应用程序接口,所以为了将现有的程序移植到某个unikernel平台,开发者
不得不根据该unikernel的API重构程序。此外,为了轻量、快速,unikernels
删去的许多基本的并且不会影响性能的安全措施,这导致unikernels相比容器
更容易受到用户程序的安全漏洞的影响。

Unikraft是一个充分考虑了兼容性和安全性的unikernel,它将系统分割成
若干相对独立的模块,各个模块可以独立安装和更新,就像传统操作系统上的动态库。
在创建系统镜像时,Unikraft提供的编译系统能够编译用户需要使用的模块,并将
它们与用户代码一起连接成可引导镜像。

我们小组计划仿照Unikraft的架构,用Rust语言编写能在RISC-V架构+ KVM平台上
运行的unikernel——Runikraft。Runikraft的核心代码使用Rust编写,但
允许用户代码使用任何语言编写——只要它能够被编译成入口为\texttt{main}的
目标代码。Runikraft强调构建系统镜像的简洁,用户只需要
修改现有的项目的编译参数就可以构建基于Runikraft的系统镜像,而不必使用
专用的工具链,更不需要重构代码。Runikraft是POSIX兼容的,所以它将支持
内存管理、进程调度,甚至磁盘管理和进程通信。不过,这些功能都是可选的且可
拓展的,如果用户不需要某项功能,他可以不将相关模块打包进系统镜像中,
如果用户能够提供某些功能的更好实现,他可以用自己的实现替换原有的模块,
甚至POSIX兼容层本身也是可选的,如果用户愿意为了效率重构代码,他也可以
直接用Runikraft的专用API。
Runikraft可以支持多进程,因为我们认为,将若干密切管理的程序打包到一个
镜像会提高效率。
\section{项目背景}


\section{立项依据}

\section{前瞻性/重要性分析}

\section{相关工作}


\end{document}